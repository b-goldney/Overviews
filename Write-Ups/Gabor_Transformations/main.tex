\documentclass{cup-pan}
\usepackage[utf8]{inputenc}
\usepackage{blindtext}
\usepackage{graphicx}
\usepackage{cite}

\title{Gábor Transorms (Short-Time Fourier Transform)}

\author[1]{\normalsize Brandon Goldney}

\affil[1]{Department of Applied Mathematics, University of Washington, Email: \url{Goldney@uw.edu}}

\begin{document}
\maketitle

\begin{abstract}
The Gábor Filter shows the frequency of a signal in the time domain.  This is an advancement from the Fourier Transform which shows a signal's amplitude in the frequency domain.  The purpose of this paper is to demonstrate the advantages and limitations of the Gábor Filter.  An 8.9 second audio clip of Handel's "Messiah" is analyzed to determine the frequencies present in the time domain.  Subsequently, a rendering of the children's classic song "Mary Had a Little Lamb" is passed through a Gábor Filter to assess the frequencies present.
\end{abstract}

\section{Section I: Introduction and Overview}
Heisenberg's Uncertainty Principle asserts that the position and velocity of an object cannot be measured exactly, at the same time, even in theory. One example of this is the Fourier Transform, which decomposes a function into its constituent frequencies. Fourier Series are powerful because they are able to represent a function as a series of $sin$ and $cos$ waves. However, the cost of performing a Fourier Transform is that all time information is discarded.  In other words, the signal which was originally in the time domain will now be represented in the frequency domain.

\vskip 0.1in
\noindent
A Fourier Transform is very good at determining the constituent frequencies of a signal, but that signal must be stationary (i.e. the frequencies are constant throughout time).  Consider a song where the frequencies change throughout time. A Fourier Transform will fail to ascertain the frequencies used in a non-stationary signal. Gábor Wavelets are designed to overcome this challenge.  Wavelets are effectively windows dragged over the signal, at each iteration a Fourier Transform is performed on the windowed section of the signal. Sliding the window across the signal highlights the Heisenberg Uncertainty Principle. The width of the window can be narrow or wide, based off the desired accuracy.  A wider window will capture more frequencies, but not localize the time well. 

\vskip 0.1in
\noindent
This leads us to the concepts of oversampling and undersampling.  Oversampling will occur when the window is too narrow.  The downside to oversampling is the Gábor Transformation will be computationally expensive, and can be prohibitively time consuming for real-world applications.  Undersampling occurs when the window is too wide.  In this scenario, the Fourier Transform is performed over a wide range of frequencies.  The multitude of frequencies will cause the Gabor Transform to become blurry because too many frequencies are being represented.  

\section{Section II: Theoretical Background}

In order to localize with respect to both time and frequency Gabor introduced a kernel:  
$$g_{t, \omega}(\tau) = e^{i \omega \tau}g(\tau - t)$$

\vskip 0.1in
\noindent
Incorporating Gabor's kernel with the Fourier Transform results in the following:  
$$G[f](t,\omega) = \tilde{f}_g(t,\omega) = \int_{-\infty}^{\infty} \space f(\tau) \bar{g}(\tau - t) \space e^{i \omega \tau} d \tau \space = \space (f, \bar g_{t,\omega})$$

\vskip 0.1in
\noindent
The bar denotes the complex conjugate.  While this formula appears intimidating at first, it can be broken down into its components to better understand it.  Notice, we're integrating over the term $\tau$, which effectively slides the window over the signal.  The new term, $\bar g (\tau - t)$ was conceptually explained earlier, it localizes the signal with respect to time and frequency.  The final term, $e^{-i \omega \tau}$, is the standard Fourier Transform.  

\vskip 0.1in
\noindent
The Gábor transform is computed by discretizing the time and frequency domain.  Consider the following sample points:  
$$ v = m \omega_0 \\ 
\tau = n t_0$$
where $m$ and $n$ are integers and $w_0$, $\tau_0$ > 0 are constants. In this case, the discrete version of Gábor's kernel becomes:
$$g_{m,n}(t) = e^{i 2 \pi m \omega_0 t} g(t-n t_0)$$
Accordingly, the Gábor Transform becomes:
$$\tilde{f}(m,n) = \int^{\infty}_{-\infty} f(t) \bar g_{m,n}(t) dt = (f,g_{m,n})$$


\section{Section III:  Algorithm Implementation and Development}

Since the process is very similar for both Handel's "Messiah" and "Mary Had a Little Lamb", the process below is generalized to be accurate to both methods. 

\vskip 0.1in
\noindent
After loading the data, the first step is to define the length of the data and then to discretize it into $n$ data points. Next, we rescale the $n$ data points by $(2 * \pi) / L$, where $L$ equals the duration of the song. L is calculated by dividing the number of data points by the sampling frequency.  We know in later steps that we will be performing a Fourier Transform, which rearranges $X$ by shifting the zero-frequency component to the center of the array.  With that in mind, we shift the $n$ data points (i.e. use np.fft.fftshift) prior to any further steps.  

\vskip 0.1in
\noindent
Now that we have the initial variables declared and created the discretized points , we can begin working with the data from the sound file by setting up a for loop.  Within the for loop, a window (e.g. Gaussian, Shannon, etc.) is created. It's important to recall that setting up the window width (i.e. \textit{t - tslide}) is important because that will impact the trade-off in accuracy between time and frequency.  Each time the data passes through the window,the signal is multiplied by the window. The Fourier Transform is taken on that new data, and then it is shifted (i.e. np.fft.fftshift) because the FFT algorithim shifts the data, and finally it's saved.  The data is saved so the spectrogram can be constructed.   To create the graphs, we plot the \textit{time} variable on the x-axis and the wavenumbers (rescaled to be $2 * \pi$ periodic) are shifted and put on the y-axis.


\section{Section IV: Computational Results}
This section lists each type of wavelet applied to the data, and three spectrograms for each wavelet.  Each spectrogram shows a different window width (i.e. wide, narrow, and standard).

\pagebreak

$$\text{Standard Wavelet}=\begin{equation}
\left\{
  \begin{aligned}
    &\text{normal window} &= \space &e^{-1*(t-\tau)^{10})} \\
    &\text{big window} &= \space &e^{-0.01*(t-\tau)^{10})} \\
    &\text{small window} &= \space &e^{-100*(t-\tau)^{10})} \\
  \end{aligned}
  \right.
\end{equation}
$$

\begin{center}
\includegraphics[
  width=18cm,
  height=13cm,
  keepaspectratio,
]{Standard_Wavelet.png}
\end{center}

\vskip 1.0in

$$\text{Gaussian Wavelet}=\begin{equation}
\left\{
  \begin{aligned}
    &\text{normal window} &= \space &e^{-1*(t-\tau)^{2})} \\
    &\text{big window} &= \space &e^{-0.01*(t-\tau)^{2})} \\
    &\text{small window} &=\space &e^{-10*(t-\tau)^{2})} \\
  \end{aligned}
  \right.
\end{equation}
$$

\begin{center}
\includegraphics[
  width=18cm,
  height=13cm,
  keepaspectratio,
]{Gaussian_Wavelet.png}
\end{center}

\vskip 1.0in

$$\text{Mexican Hat Wavelet}=\begin{equation}
\left\{
  \begin{aligned}
    &\text{normal window} &= \space &(1-(t - \tau)^{2}* e^{-1*(t-\tau)^{2}} \\
    &\text{big window} &= \space &((1-0.1)*(t - \tau)^{2})* e^{-0.1*(t-\tau)^{2}} \\
    &\text{small window}  &= \space &((1-10)*(t - \tau)^{2})* e^{-10*(t-\tau)^{2}} \\
  \end{aligned}
  \right.
\end{equation}
$$

\begin{center}
\includegraphics[
  width=18cm,
  height=13cm,
  keepaspectratio,
]{Mexican_Hat_Wavelet.png}
\end{center}

\vskip 1.0in

$$\text{Shannon Function Wavelet}=\begin{equation}
\left\{
  \begin{aligned}
    &\text{normal window} &=  \space &|t-\tau| \leq 0.5 \\
    &\text{big window} &= \space &0.1*|t-\tau| \leq 0.5 \\
    &\text{small window}  &= \space &10*|t-\tau| \leq 0.5 \\
  \end{aligned}
  \right.
\end{equation}
$$

\begin{center}
\includegraphics[
  width=18cm,
  height=13cm,
  keepaspectratio,
]{Shannon_Fcn_Wavelet.png}
\end{center}

\textbf{Part 2: Mary Had a Little Lamb}

\begin{center}
\includegraphics[
  width=13cm,
  height=10cm,
  keepaspectratio,
]{"MHLL - Piano".png}
\end{center}

\begin{center}
\includegraphics[
  width=13cm,
  height=10cm,
  keepaspectratio,
]{"MHLL - Recorder".png}
\end{center}


\section{Section V: Summary and Conclusions}
Recall, as we increase the width of the window, we will capture a broader range of frequencies; however, we will not be able to localize the frequencies with respect to time.  The reverse is true for narrow windows.  That being said, in graphs with larger windows, we can trust the y-axis (i.e. frequency) more, and we lose faith in the x-axis (i.e time).  

\vskip 0.1in
\noindent
Graphically, we can see this in the spectrograms.  Graphs with wider windows tend to blur more in the horizontal direction, since larger windows do not localize well.  The Gaussian Wavelet, Mexican Hat, and Shannon Function clearly demonstrate the difference between large and small window sizes.  In those functions, we can see dark spaces between the color on the x-axis. Presumably this is where there are slight pauses in music.  In Handel's Messiah, we can clearly see a higher and lower frequency, presumably the difference between men and women's singing. 

\vskip 0.1in
\noindent
\textbf{Mary Had a Little Lamb}
Creating spectrograms for a piano and a recorder creates an interesting plot.  Pianos create overtones; whereas, recorders create single frequencies.  We can see more color in the y-axis direction for the piano than the recorder.   

\section{Appendix A: Python functions used and brief implementation explanation}
\begin{itemize}
\item \textbf{np.load}: Load arrays or pickled objects   
\item \textbf{sd.play}: Assuming you have a NumPy array named myarray holding audio data with a sampling frequency of fs (in the most cases this will be 44100 or 48000 frames per second), you can play it back with play()  
\item \textbf{IPython.display import Image}: Loads images from local source  
\item \textbf{Matplotlib}: Matplotlib is a Python 2D plotting library which produces publication quality figures in a variety of hardcopy formats and interactive environments across platforms  
\item \textbf{np.linspace}: Return evenly spaced numbers over a specified interval  
\item \textbf{np.fft.fftshift}: Shift the zero-frequency component to the center of the spectrum  
\item \textbf{np.fft.ffti}: Compute the one-dimensional discrete Fourier Transform  
\item \textbf{np.argmax}: Returns the indices of the maximum values along an axis  
\item \textbf{np.colormesh}: Create a pseudocolor plot with a non-regular rectangular grid (good for creating spectrograms)
\end{itemize}

\section{Appendix B: Python code}

\end{document}
