\documentclass[12pt]{article}

\usepackage[utf8]{inputenc}
% This is a package to accept utf8 input.  I normally do not use it in my documents, but it was here by default in Overleaf.
\usepackage{pgfplots}
\usepackage{amsmath}
\usepackage{amssymb}
\usepackage{amsthm}
% These three packages are from the American Mathematical Society and includes all of the important symbols and operations 
\usepackage{fullpage}
% By default, an article has some vary large margins to fit the smaller page format.  This allows us to use more standard margins.

\setlength{\parskip}{1em}
% This gives us a full line break when we write a new paragraph

\usepackage{parskip}
% This stops new paragraphs from being indented
\usepackage{xcolor}
\begin{document}

\section{Separation of Variables}

The method separation of variables relies upon the assumption that a function fo the form:
\begin{flalign}
	u(x,t) = \phi(x) \, G(t) \label{eq 1}
\end{flalign}

will be a solution to a linear homogeneous PDE. Let's look at an example.

\textbf{Example 1}
\begin{flalign}
	\dfrac{\partial u}{\partial t} = K \dfrac{\partial ^2 u}{\partial x^2} \; \; \; \;
		u(x,o) = f(x), \; \; \; \; u(0,t) = u(L,t) = 0 \label{eq 2}
\end{flalign}

Let's plug \eqref{eq 1} into the given equation.
\begin{flalign}
	\dfrac{\partial}{\partial t} \phi(x)\,  G(t) &= K \dfrac{\partial ^2}{\partial x^2} \, \phi(x) G(t) \\
	\phi(x) \frac{d}{dt} \, G(t) &= K G(t) \frac{d^2}{dx^2} \, \phi(x) \label{eq 4}
\end{flalign}

In equation \eqref{eq 4}, we factored out terms not relvant to the derivative, transforming it from
partial derivatives to normal derivatives. 

At this point, we want all the $t$'s on one side and all the $x$'s on the other side.
Dividing by $\phi(x) \, G(t)$, produces the desired result. 
\begin{flalign}
	\dfrac{1}{G} \, \dfrac{dG}{dt} &= K \dfrac{1}{\phi} \; \dfrac{d^2 \phi}{dx^2} \\
	\dfrac{1}{K\, G} \, \dfrac{dG}{dt} &= \dfrac{1}{\phi} \; \dfrac{d^2 \phi}{dx^2} \label{eq 6}
\end{flalign}
In equation \eqref{eq 6}, we're diving by $K$, this is done for convenience down the road. 

For these to be equal, given they have different variables, they must both equal the same constant.
Therefore, we can write:
\begin{flalign}
	\dfrac{1}{KG} \, \dfrac{dG}{dt} = \dfrac{1}{\phi} \, \dfrac{d^2 \phi}{dx^2} = - \lambda \label{eq 7}
\end{flalign}
Let's rewrite \eqref{eq 7} as two ODEs:
\begin{flalign}
	\dfrac{dG}{dt} \; &= \; - \lambda KG \label{eq 8} \\
	\dfrac{d^2 \phi}{dx^2} \; &= \; -\lambda \phi \label{eq 9}
\end{flalign}

Let's apply the left BC: $u(0,t) = 0$
\begin{flalign}
	u(0,t) = \phi(0) \, G(t) = 0
\end{flalign}
Recall the trivial solution is when $u(x,t) = 0$. To avoid this trivial solution, we have to set $\phi(0)=0$.
Otherwise, $G(t)$ would have to be 0 for all $t$, resulting in the trivial solution. 

Apply the right BC: $u(L,t) = 0$ 
\begin{flalign}
	u(L,t) = \phi(L) \, G(t) = 0 \label{eq 11}
\end{flalign}
Using similar logic as for the left BC to avoid the trivial solution, we set $\phi(L) = 0$. 

Recapping what we've done so far, we've transformed the given problem into two ODEs.  One ODE is to solve for 
the time derivative, and the other ODE is to solve the spatial derivative. Additionally, we were able to 
infer that both boundary conditions equal 0. \\*
\linebreak

\textbf{Example 2}
\begin{flalign}
	\dfrac{\partial u}{\partial t} = K \dfrac{\partial ^2 u}{\partial x^2} \; \; \; \;
		u(x,o) = f(x), \; \; \; \frac{\partial u}{\partial x}(0,t) \, = \, 
		\frac{\partial u}{\partial x} (L,t) = 0
\end{flalign}
This is an example with no sources and perfectly insulated boundaries. So, let's assume a solution of the 
following form.
\begin{flalign}
	u(x,t) = \phi(x) \, G(t) \label{2.1}
\end{flalign}
Similar to the prior problem, let's plug the given equation into \eqref{2.1}.
\begin{flalign}
	\dfrac{\partial}{\partial t} \phi(x) G(t) &= K \dfrac{\partial ^2}{\partial x^2} \, \phi(x) G(t) \\
	\phi(x) \dfrac{d}{dt} G(t) &= K \, G(T) \, \dfrac{d^2}{dx^2} \, \phi(x) \label{eq 2.3} \\
	 \dfrac{1}{K \, G} \dfrac{d}{dt} G(t) &= \dfrac{1}{\phi(x)} \, \dfrac{d^2}{dx^2} \, \phi(x) \\
	&= - \lambda \label{eq 2.4}
\end{flalign}
In equation \eqref{eq 2.3} we factored out the constant so we could rewrite it as a normal derivative. 
In equation \eqref{eq 2.4}, we divided by $\phi(x) G(t)$ and $K$ for conveneience, and set it equal to 
a constant $-\lambda$. 

Let's rewrite \eqref{eq 2.4} as two ODEs.
\begin{flalign}
	\dfrac{d}{dt} G = -\lambda KG \\
	\dfrac{d^2}{dx^2} \phi = -\lambda \phi
\end{flalign}

Let's check the BCs, starting with the left BC
\begin{flalign}
	&\dfrac{\partial (G(t) \, \phi(x))}{\partial x} (0,t) = 0 \\
\end{flalign}
To avoid the trivial solution, we'll set $\frac{d}{dx} \phi(0) = 0$. 

Apply the right BC:
\begin{flalign}
	\dfrac{\partial (G(t) \, \phi(x))}{\partial x} (L,t) = 0 \\
	G(t) \, \dfrac{d}{dx} \, \phi(L) = 0
\end{flalign}

To avoid the trivial solution, we'll set $\dfrac{d}{dx} \phi(L) = 0$. 

Let's recap what we found. We transformed the PDE to two ODEs, and found the derivatives of both BCs are equal to 0. 
\begin{flalign}
	\dfrac{d}{dx} \phi(0) = 0 \\
	\dfrac{d}{dx} \phi(L) = 0
\end{flalign}

\pagebreak[4]
\section{Solving the Heat Equation}

Let's look at the given problem.
\begin{flalign}
	\dfrac{\partial u}{\partial t} = K \dfrac{\partial ^2 u}{\partial x^2} \; \; \; \;
		u(x,o) = f(x), \; \; \; u(0,t) = u(L,t) = 0
\end{flalign}

We've seen this problem in the prior section.  The two ODEs we need are:
\begin{flalign}
	\dfrac{dG}{dt} = -K \lambda G \\
	\dfrac{d^2 \phi}{dx^2} + \phi \lambda = 0
\end{flalign}
and the BCs are: $\phi(0) = 0$ and $\phi(L) = 0$. 

At this point, we have two ODEs to solve. 

Let's solve the spatial differential equation first.  Notice, the spatial differential equation is very similar to the first
example in the eigenvalues and eigenfunctions section (i.e. $y'' + \lambda y = 0$).  So, following a similar method from that
section, we'll solve for three different cases of $\lambda$: $\lambda > 0, \lambda = 0 $, and $\lambda < 0$. 

\textbf{Case 1: $\lambda > 0$}
\begin{flalign}
	r^2 + \lambda = 0 \\
	r_{1,2} = \pm \sqrt{\lambda} \; i
\end{flalign}

The general solution for complex roots is:
\begin{flalign}
	\phi(x) = c_1 e^{a \, x} \, cos(ux) + c_2 e^{a \, x} \,sin(ux)
\end{flalign}
Where the roots are written as: $r_{1,2} = a \pm u \, i$.  So, in this case, the $e^{a \, x}$ term falls away 
because $a = 0$. 

Apply the first BC: $u(0, t) = 0$ 
\begin{flalign}
	\phi(0) &=  c_1 \, cos(0) + c_2 sin(0) \\
	&\implies 0 = c_1
\end{flalign}

Apply the second BC: $u(L,t) = 0$
\begin{flalign}
	\phi(L) = c_2 \, sin(\sqrt{\lambda} \, L) 
\end{flalign}
In the above equation, $c_1 cos(ut)$ falls away because $c_1 = 0$. Since we're after the non-trivial solution,
we want:
\begin{flalign}
	&sin(\sqrt{\lambda} \, L) = 0 \\
	&\implies \sqrt{\lambda} \, L = n \pi \; \; \; \; \forall \; \; n = 1,2,3,... \\
	&\lambda = \left(\dfrac{n \pi}{L} \right)^2
\end{flalign}

Plugging this into the formla for the general solution, we get the following:
\begin{flalign}
	\phi(x) = sin \left(\dfrac{n \pi x}{L}\right) \; \; \; \; \forall \; n = 1,2,3,...
\end{flalign}
Notice the $c_2$ term is not written explicity - that's because we'll absorb it into another
constant later.

\textbf{Case 2: $\lambda = 0$}
\begin{flalign}
	0 &= r^2 + \lambda \\
	r^2 &= 0
\end{flalign}
Integrating twice results in(we're integrating because we need the value of $r$, not $r^2$):
\begin{flalign}
	r  = c_1 + c_2x
\end{flalign}
Applying the first BC: $\phi(0) = 0$
\begin{flalign}
	0 = \phi(0) = c_1
\end{flalign}
Apply the second BC: $\phi(L) = 0$
\begin{flalign}
	0 &= \phi(L) = c_2L \\
	&\implies c_2 = 0
\end{flalign}
Takeaway: only the trivial solution is possible in this case

\textbf{Case 3: $\lambda < 0 $}
\begin{flalign}
	0 &= r^2 - \lambda \\
	r &= \pm \sqrt{\lambda}
\end{flalign}
Recall the usual general solution for real distinct roots is:
\begin{flalign}
	\phi(x) = c_1e^{r_1x} + c_2 e^{r_2x}
\end{flalign}
but we can rewrite this in hyperbolic form:
\begin{flalign}
	\phi(x) = c_1 cosh(a \, x) + c_2 sinh(a \, x)
\end{flalign}
where $r_1 = a$ and $r_2 = a$.  Therefore, we can write the following:
\begin{flalign}
	\phi(x) = c_1 \, cosh(\sqrt{- \lambda} x) + c_2 sinh(\sqrt{- \lambda} x)
\end{flalign}

Let's apply the first BC:
\begin{flalign}
	0 &= \phi(0) = c_1 \, cos(0) + 0 \\
	&\implies c_1 = 0
\end{flalign}
Applying the second BC:
\begin{flalign}
	0 = \phi(L) = c_2 \, sinh(\sqrt{-\lambda} \, L)
\end{flalign}
To avoid the trivial solution we need to avoid setting $c_2 = 0$.  We are already assuming that $\lambda < 0$ and that means
$L \, \sqrt{-\lambda} \ne 0$, meaning $sinh(L \, \sqrt{-\lambda}) \ne 0$. Therefore, we must have $c_2 = 0$, resulting 
in the trivial solution.

Finally, the complete list of eigenvalues and eigenfunctions for the spatial differnetial equation are:
\begin{flalign}
	\lambda _n &= \left(\dfrac{n \pi}{L}\right)^2 \\
	\phi(x) &= sin\left( \dfrac{n \pi x}{L}\right) \; \; \; \; \forall \; n=1,2,3,...
\end{flalign}

Now, let's solve the time differential equation.
\begin{flalign}
	\dfrac{dG}{dt} = -K \lambda _n G
\end{flalign}
This is a first order, linear, separable differential equation with the following solution:
\begin{flalign}
	G(t) &= ce^{-K \lambda_n t)} \\
	&= ce^{-K (\frac{n \pi}{L})^2 \, t}
\end{flalign}
In the last equation, we're plugging in $\lambda = (\frac{n \pi}{L})^2$  

We've solved both the differential equations so we can write down a solution.
\begin{flalign}
	u_n(x,t) &= \phi(x) \, G(t) \\
	&= B_n \, sin\left(\dfrac{n \pi x}{L} \right) \, e^{-K \, (\frac{n \pi}{L})^2 \, t} \; \; \; \; \forall \; n=1,2,3,...
\end{flalign}
We changed $c$ in the solution to the time differential equation to $b_n$ to denote it will probably be different
for each value of $n$.  

\end{document}
