\documentclass[12pt]{article}
  
\usepackage[utf8]{inputenc}
% This is a package to accept utf8 input.  I normally do not use it in my documents, but it was here by default in Overleaf.
\usepackage{pgfplots}
\usepackage{amsmath}
\usepackage{amssymb}
\usepackage{amsthm}
% These three packages are from the American Mathematical Society and includes all of the important symbols and operations 
\usepackage{fullpage}
% By default, an article has some vary large margins to fit the smaller page format.  This allows us to use more standard margins.

\setlength{\parskip}{1em}
% This gives us a full line break when we write a new paragraph

\usepackage{parskip}
% This stops new paragraphs from being indented
\usepackage{xcolor}

\begin{document}

\section{Eigenvalues and Eigenfunction} \\*
Recall, for a given square matrix, $A$, with an eigenvalue of $\lambda$ and eigenvector of
$\overrightarrow{x}$, we can write: 
\begin{flalign}
A \overrightarrow{x} \, = \, \lambda \overrightarrow{x}
\end{flalign}

In order for $\lambda$ to be an eigenvalue then we had to be able to find non-zero solutions to the equation. \\*

This raises the question, how do eigenvalues connect to boundary value problems (BVPs)? Values of $\lambda$ 
that produce a non-trivial solution will be referred to as eigenvalues, and the non-trivial solutions wil be
called eigenfunctions.  \\*


Let's consider an equation of the following form:

\begin{flalign}
	y'' \, + \lambda y = 0
\end{flalign}
For those values of $\lambda$ that produce a non-trivial solution we will call that an eigenvalue, 
and the solution itself will be referred to as the eigensolution. \\*

Let's look at a quick example. \\*

\textbf{Example 1: } \\*
Solve the BVP
\begin{flalign}
	y''+4y=0 \; \; \; y(0)=0, \; \; y(\lambda \pi)=0
\end{flalign}

The characteristic equation is:
\begin{flalign}
	0 &= r^2+4 \\
	r &= 2i
\end{flalign}

The general solution to complex roots is: 
\begin{flalign}
	y(x) = c_1e^{\lambda \, x}cos(ux) + c_2e^{\lambda \, x} sin(ux)
\end{flalign}
where the roots are defined as $r_{1,2} \, = \, \lambda \pm u \, i$; so, $\lambda = 0$ and $u=2$ 
(because $r \, = \, 0 \pm 2 \, i$).  
Thefore, we can write:

\begin{flalign}
	y(x) \, = \, c_1 \, cos(2x) + c_2 \, sin(2x)
\end{flalign}

Let's find $c_1$ and $c_2$ by plugging-in the boundary conditions (BCs). \\*

Apply the first BC: $y(0)=0$
\begin{flalign}
	0 = y(0) &= c_1 \, cos(0) + c_2 \, sin(0) \\
	&= c_1 \,1 + c_2\,0 \\
	&\implies c_1 = 0
\end{flalign}


Apply the second BC: $y(2 \pi) = 0$
\begin{flalign}
	0 =y(2\pi) &= c_1 \, cos(4 \pi) + c_2 \, sin(4 \pi) \\
	&= c_1 + 0 \\
	&\implies c_1 = 0
\end{flalign}

Therefore, $c_2$ is arbitrary and the solution is:
\begin{flalign}
	y(x) = c_2 \, sin(2x)
\end{flalign}

In this example, $\lambda = 4$, and we found a non-trivial solution. Therefore, the eigenvalue is 4, and
the eigensolution is $y(x) = c_2 \, sin(2x)$.

\textbf{Example 2: } \\*
Find the eigenvalues and eigensolutions for the BVP.

\begin{flalign}
	y'' \, + \, 3y = 0 \; \; \; \; \; y(0) = 0, \; \; y(2 \pi) = 0
\end{flalign}

The characteristic equation is:
\begin{flalign}
	0 &= r^2 \, + \, 3 \\
	r & = \sqrt{3} \, i
\end{flalign}

The general solution is:
\begin{flalign}
	y(x) \, &= \, c_1 e^{\lambda x} cos(ux) \, + \, c_2 e^{\lambda x} sin(ux)  \\
	&= c_1 cos(\sqrt{3} \, x) + c_2 sin(\sqrt{3} \, t)
\end{flalign}

Let's apply the BCs. \\*

Apply the first BC: $y(0) = 0$ \\*
\begin{flalign}
	0 = y(0) = c_1 \\
	\implies c_1 = 0
\end{flalign}

Apply the second BC: $y(2 \pi) = 0$
\begin{flalign}
	0 = y(2 \pi) &= 0 + c_2 sin(\sqrt{3} \, 2 \pi) \\
	&= 0 + c_2(-0.9936) \\
	&\implies c_2 = 0
\end{flalign}

Since $c_1 = c_2 = 0$ the solution is $y(x) = 0$, which is obviously trivial.  Therefore, 
there are no eigenvalues or eigensolutions to this equation. \\*

At this point, we've seen two examples, from the same generalized formula: $y'' + \lambda y = 0$ with 
the same BCs.  Let's find all the eigenvalues and eigensolutions to this formula.  \\*

\textbf{Example 3} \\*
Let's generalize the previous two examples by looking at the following formula:
\begin{align}
	y'' \, + \, \lambda y = 0
\end{align}

We'll solve this problem by looking at the following three cases: i) $\lambda > 0 \;$, ii) $\lambda = 0 \;$,
and iii) $\lambda < 0 \;$. \\*

\textbf{Case 1: $\lambda > 0$} \\*

In this case, the characteristic equation is: 
\begin{align}
	0 &= r^2 + \lambda \\
	r_{1,2} &= \pm \sqrt{- \lambda}
\end{align}

Since we're assuming $\lambda > 0$, we can write:
\begin{align}
	r_{1,2} = \pm \sqrt{\lambda} \, i
\end{align}

The general solution is: 
\begin{align}
	y(x) = c_1 cos(\sqrt{\lambda} \, x) + c_2 sin(\sqrt{\lambda} \, x) \\
\end{align}

Apply the first BC: $y(0) = 0$ \\*
\begin{align}
	0 = y(0) &= c_1 \\
	\implies c_1 = 0
\end{algin}

Apply the second BC: $y(2 \pi) = 0$ \\*
\begin{align}
	0 = y(2 \pi) &= c_2 sin(2 \pi \sqrt{\lambda})
\end{align}

This means either $c_2 = 0$ or $sin(2 \pi \sqrt{\lambda})=0$. Remember, we want non-trivial solutions, 
and if $c_2 = 0$ then that would be the trivial solution.  Therefore, let's assume $c_2 \ne 0$, resulting in:
\begin{align}
	0 &= sin(2 \pi \sqrt{\lambda}) \\
	&\implies 2 \pi \sqrt{\lambda} = n \pi \; \; \; \; \forall n \, = \, 1,2,3,... \\
	&\implies \lambda = \dfrac{n^2}{4}
\end{align}

We're taking advantage of the fact that we know where $sin \, = \, 0$ and that $\lambda = 0$ results in
$2 \pi \sqrt{\lambda} > 0$. \\*

The corresponding eigenvalues and eigenfunctions are:
\begin{align}
	&\lambda = \dfrac{n^2}{4} \; \; \; \; \forall \; \; \; \; n = 1,2,3,... \\
	&y_n(x) = sin\left(\dfrac{nx}{2}\right) 
\end{align}

\textbf{Case 2: $\lambda = 0$} \\*
In this case, the BVP becomes;
\begin{align}
	y''=0 \;, \; \; \; y(0) = 0 \;, \; \; \; y(2 \pi) = 0
\end{align}

Integrating twice results in:
\begin{align}
	y' &= c_1 \\
	y &= c_1x + c_2
\end{align}

Apply the first BC: $y(0) = 0$ 
\begin{align}
	0 = y(0) = c_2
\end{align}

Apply the second BC: $y(2 \pi) = 0$
\begin{align}
	0 = y(2 \pi) &= c_1 2 \pi + c_2 \\
	&\implies c_1 = 0
\end{align}

Since $c_2$ is already 0, then $c_1$ must be 0 resulting in the trivial solution. 

\textbf{Case 3: $\lambda < 0$} \\*
The characteristic equation is:
\begin{align}
	0 &= r^2 - \lambda  \\
	r_{1,2} &= \pm \sqrt{\lambda}
\end{align}
 
The usual general solutions for a differential equation with two distinct real roots is:
\begin{align}
	y(x) = c_1 e^{r_1 x} + c_2 e^{r_2 x}
\end{align}
but it's more convenient to rewrite that in hyperbolic form:
\begin{align}
	y(x) = c_1 cosh(\lambda x) + c_2 sinh(\lambda x)
\end{align}

Apply the first BC: $y(0) = 0$ \\*
\begin{align}
	0 = y(0) &= c_1 cosh(0) + c_2 sinh(0) \\
	&= c_1 \\
	&\implies c_1 = 0
\end{align}

Apply the second BC: $y(2 \pi) = 0$ 
\begin{align}
	0 = y(2 \pi) &= c_2 sinh(2 \pi \sqrt{\lambda})
\end{align}

Since we know that $sinh(2 \pi \sqrt{\lambda}) \ne 0$ we know that $c_2 = 0$. \\*

Takeaway: for this BVP, when $\lambda < 0$, we only have the trivial solution. 

\textbf{Summary: } For all three cases, we have the following eigenvalues and eigensolutions:
\begin{align}
	\lambda _n &= \dfrac{n^2}{4} \\
	 y_n(x) &= sin\left(\dfrac{nx}{2}\right) \; \; \; \; \forall n=1,2,3,...
\end{align}





























\end{document}
